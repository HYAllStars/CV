\documentclass{letter}
\usepackage{hyperref}
\signature{Hyeyoung Shin}
\usepackage[left=3cm, right=3cm, top=1.5cm, bottom=1.5cm]{geometry}

%% \documentclass{article}
%% \usepackage[utf8]{inputenc}


\address{%
Hyeyoung Shin\\
1102 Carroll Avenue\\
Ames, IA 50010, USA\\
(212)535-8565}
%% hyeyoungshinw@gmail.com}
\begin{document}
 
\begin{letter}{%
    Dr.~Hridesh Rajan\\
    Laboratory of Software Design\\
    Department of Computer Science\\
    Iowa State University \\
    Ames, IA 50011}
\date{\today}
\opening{Dear Dr.~Rajan,}

I am writing to you regarding a PhD position for the Specification Inference Project,
and I would like to take this opportunity to explain why and how I feel I
could contribute to the project.

First, I have taken and excelled in the courses in programming languages (PL)
and formal methods, and these courses provide necessary background for the project.
Second, my research interests and aspirations continue to motivate me to be involved in
and make valuable contributions to projects at the intersection of PL
theory and practice.
Third, Java is the programming language with which I am most comfortable,
but I also have extensive experience with Coq, and I am quite comfortable with
collaboration tools like git and GitHub.
I am also proficient at functional programming in languages like SML, Racket,
and Scala.%% \marginpar{Maybe say why Agda and Haskell or leave out this sentence?}
%% In the near future, I expect to gain proficiency in other languages like Agda and Haskell.


I have attended several PL conferences and workshops and
have learned that specification and verification are at the heart of 
programming languages and software engineering research.
When I read the paper entitled
``Inferring Behavioral Specifications from Large-scale Repositories
by Leveraging Collective Intelligence''
(hereinafter IBS), I was excited by the idea of using automatically generated
specifications to make verified programming easier for developers.
Especially with BOA's large scale knowledge-base, the approach taken in the paper seems
powerful and interesting.

I think we could exploit the method of specifying imperative programs in Hoare Type Theory
in Coq to generate safe and tight specifications for widely-used code. I would like to
continue to pursue this direction further and develop similarity, differential, and
decomposition-based techniques, using methods that take advantage of programming analysis
and software repository mining to avoid the shortcomings of other approaches.
To do this, I intend to learn more about BOA and related prior work on
specification inference, including the articles cited in the IBS paper.
Moreover, I would like to apply some of the powerful theories and tools that I learned
at the Oregon Programming Languages Summer School to prove soundness of our strategy.

Naturally, as you are the project PI, I would defer to you and seek your
expert guidance regarding research goals, action items, deliverables, etc.
I could commit to providing regular status updates, and engage in
frequent email correspondence with you and other team members.
I have already met or taken classes with a number of them, and I expect it
will be easy to establish a good rapport with the group.

Please accept the enclosed r\'{e}sum\'{e} and feel free to contact me with any
questions. I appreciate your time and consideration.

\closing{Sincerely,}
\encl{r\'{e}sum\'{e}}

\end{letter}

\end{document}
